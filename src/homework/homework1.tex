%! Author = Len Washington III
%! Date = 8/19/24

% Preamble
\documentclass[
	% author={}, % Remove the percentage sign and add your name within the brackets.
	number={1},
	title={Review of Matrices}
]{math486homework}

% Document
\begin{document}

\maketitle

\noindent Review solving linear equations using matrices in any linear algebra text or check out Paul Dawkins' linear algebra notes that are linked on the homework page.

For problems 1 and 2, convert the system of equations into a matrix problem $A\vec{x}=\vec{b}$ by defining $A$ and $\vec{b}$.
Define an augmented matrix $(A|b)$.
Solve using Gaussian elimination to convert $A$ into upper triangular or row-echelon form.
Show all steps.

\begin{problems}
	\problem Consider the linear system:
	\begin{equation*}
	\begin{aligned}
		 x_{1} -  x_{2} -  x_{3} &= -3\\
		2x_{1} + 3x_{2} + 5x_{3} &= 7\\
		 x_{1} - 2x_{2} + 3x_{3} &= -11\\
	\end{aligned}
	\end{equation*}
	\addanswer{Problem-1}
	\problem Consider the linear system:
	\begin{equation*}
	\begin{aligned}
		 x_{1} - x_{2} - x_{3} &= 8\\
		 x_{1} - x_{2} + x_{3} &= 3\\
		-x_{1} + x_{2} + x_{3} &= 4\\
	\end{aligned}
	\end{equation*}
	\addanswer{Problem-2}
	\problem Consider the linear system:
	\begin{equation*}
	\begin{aligned}
		kx_{1} + 6x_{3} &= 51\\
		12x_{2} - 6x_{3} &= -6\\
		x_{1} - x_{2} - x_{3} &= 0\\
	\end{aligned}
	\end{equation*}
	\begin{problems}
		\subproblem Write the system in matrix form $A\vec{x}=\vec{b}$ by defining $A$ and $\vec{b}$. \addanswer{Problem-3a}
		\subproblem For what values of $k$ does the system have a unique solution and when does it not have an inverse? \addanswer{Problem-3b}
	\end{problems}
	\problem Consider $A\vec{x}=\vec{b}$ where
	\[ A = \left[ \begin{array}{rrrr}
		1 & 2 & -3 & 1\\
		-1 & -1 & 4 & -1\\
		-2 & -4 & 7 & -1\\
	\end{array} \right],\ \ \ 
	\vec{b} = \left[ \begin{array}{c}
		1\\
		6\\
		1\\
	\end{array} \right].\]
	\begin{problems}
		\subproblem Use Gaussian elimination with an augmented matrix $(A|b)$ to solve for $\vec{x}$. \addanswer{Problem-4a}
		\subproblem Solve the associated homogeneous equation $A\vec{x}=0$. \addanswer{Problem-4b}
		\subproblem What is the dimension of the kernel $K(A)$ or nullspace $N(A)$? \addanswer{Problem-4c}
	\end{problems}
	\problem Consider $A\vec{x}=\vec{b}$ where
	\[ A = \left[ \begin{array}{rrrr}
		1 & 3 & 1 & 1\\
		2 & -2 & 1 & 2\\
		1 & -5 & 0 & 1\\
	\end{array} \right],\ \ \
	\vec{b} = \left[ \begin{array}{c}
		3\\
		8\\
		5\\
	\end{array} \right].\]
	\begin{problems}
		\subproblem Use Gaussian elimination with an augmented matrix $(A|b)$ to solve for $\vec{x}$. \addanswer{Problem-5a}
		\subproblem Solve the associated homogeneous equation $A\vec{x}=0$. \addanswer{Problem-5b}
		\subproblem What is the dimension of the kernel $K(A)$ or nullspace $N(A)$? \addanswer{Problem-5c}
	\end{problems}
	\problem Find the kernel or nullspace of:
	\[ A = \left[ \begin{array}{cccc}
		1 & 1 & 1 & 0\\
		2 & 1 & 0 & 1\\
	\end{array} \right] \]
	\addanswer{Problem-6}
	\problem The velocity $v$ at which flow in a pipe will switch from laminar to turbulent depends on the diameter $d$ of the pipe as well as the density $\rho$ and the dynamic viscosity $\mu$ of the fluid.
	\begin{problems}
		\subproblem Find a dimensionally reduced form for $v$. \addanswer{Problem-7a}
		\subproblem Suppose the pipe has diameter $d=100$ and for water (where $\rho=1$ and $\mu=10^{-2}$) it is found that $v=0.25$.
		What is the $v$ for the olive oil (where $\rho=1$ and $\mu=1$).
		The units here are in cgs. \addanswer{Problem-7b}
	\end{problems}
\end{problems}

\end{document}