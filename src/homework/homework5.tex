%! Author = Len Washington III
%! Date = 10/2/24

% Preamble
\documentclass[
	% author={}, % Remove the percentage sign and add your name within the brackets.
	number={5},
]{math486homework}

% Document
\begin{document}

\maketitle

\begin{problems}
	\problem Strep throat, sinus infections, etc. usually require an antibiotic to help bring the infection under control.
	Zithromax (azithromycin) is often prescribed for these infections as a Z Pak containing 6 pills.
	Suppose we design a two compartment model for Zithromax with the first compartment being the GI tract $x(t)$ and the second compartment being the blood stream, $y(t)$ with the following system of ODE's:
	\begin{equation*}
	\begin{aligned}
		x' &= -k_{1}x + I(t)\\
		y' &= k_{1}x - k_{2}y\\
	\end{aligned}
	\end{equation*}
	where $I(t)$ is the input of the pills.
	The initial amount in each compartment equal to 0.
	The dosing regimen for a Z Pak is 2 pills the first day and then 1 pill for the following 4 days (5 day regimen).
	The time between the doses is 1 day and each pill delivers $D$ units of the drug.
	\begin{problems}
		\subproblem Find the amount of the drug in each compartment from days 1 to 8.
		Model each pill dose by a Dirac delta function spiked at the appropriate time. \addanswer{Problem-1a}
		\subproblem If each pill is 400mg, $k_{1} = 0.9$, and half-life of the drug in the blood is 2.3 days, graph $x(t)$ and $y(t)$ on the same axes from day 1 to day 8. \addanswer{Problem-1b}
	\end{problems}
	\problem Consider a system of ODE's with initial conditions.
	\begin{equation}
		\frac{d\vec{x}}{dt} = A\vec{x} + \vec{b} = \left[ \begin{array}{cc}
			-2 & 1\\
			1 & -2
		\end{array} \right]\vec{x} + \left[ \begin{array}{c}
			3\\
			-1
		\end{array} \right], \vec{x}(0) = \left[ \begin{array}{c}
			2\\
			2
		\end{array} \right]
		\label{eq:1}
	\end{equation}
	\begin{problems}
		\subproblem \addanswer{Problem-2a}
		\subproblem \addanswer{Problem-2b}
		\subproblem \addanswer{Problem-2c}
		\subproblem \addanswer{Problem-2d}
		\subproblem
		\begin{problems}
			\subsubproblem \addanswer{Problem-2e-i}
			\subsubproblem \addanswer{Problem-2e-ii}
			\subsubproblem \addanswer{Problem-2e-iii}
		\end{problems}
	\end{problems}
	\problem
	\begin{problems}
		\subproblem \addanswer{Problem-3a}
		\subproblem \addanswer{Problem-3b}
		\begin{problems}
			\subsubproblem \addanswer{Problem-3b-i}
			\subsubproblem \addanswer{Problem-3b-ii}
			\subsubproblem \addanswer{Problem-3b-iii}
		\end{problems}
		\subproblem \addanswer{Problem-3c}
	\end{problems}
\end{problems}

\end{document}