%! Author = Len Washington III
%! Date = 9/04/24

% Preamble
\documentclass[
	date={September 4{,} 2024}
]{math486notes}
\definecolor{emphorange}{HTML}{ed3624}

% Document
\begin{document}

Let $c(x,t) = $ concentration of a chemical in a region $(x, x+dx)$ at time $t$.

In 3 dimensions: $[c(x,t)] = ML^{-3}$.

In 1 dimension: $[c(x,t)] = ML^{-1}$.

Key: $c(x,t)$ is measurable

\begin{equation*}
\begin{aligned}
	\int_{a}^{b} c(x,t)dx = \mbox{ total concentration of chemical in } (a, b) \mbox{ at time }t
\end{aligned}
\end{equation*}
Check the units
\begin{equation*}
\begin{aligned}
	\left[ \int_{a}^{b} c(x,t)dx \right] = ML^{-1}\times L = M
\end{aligned}
\end{equation*}
How does a chemical move?
\definition{flux $J(x,t)$}{amount of substance that passes through $x$ in the positive direction per unit time.}

The rate of change of the amount is
\[ \frac{d}{dt}\int_{a}^{b} c(x,t)dx = J(a,t) - J(b,t) \]
or
\[ \int_{a}^{b} \frac{\partial c}{\partial t}dx = -\int_{a}^{b} \frac{\partial J}{\partial x} \Rightarrow \int_{a}^{b} \left( \frac{\partial c}{\partial t} + \frac{\partial J}{\partial x} \right)dx = 0 \]
This is the \textbf{Conservation of Mass!}
\begin{equation}
	\frac{\partial c}{\partial t} + \frac{\partial J}{\partial x} = 0
	\label{eq:conservation-of-chemical-mass}
\end{equation}

What is the flux $J$?

How is it related to $c(x,t)$?

\section{Chemical Diffusion}\label{sec:chemical-diffusion}
\definition{Fick's Law}{chemical moves from regions of higher concentration to lower concentration.}
\[ J \propto -\frac{\partial c}{\partial x} \]
or
\[ J = -D\frac{\partial c}{\partial x} \]
where $D$ is the diffusion constant.

Substitute $J=-D\frac{\partial c}{\partial d}$ into~\eqref{eq:conservation-of-chemical-mass}
\begin{equation}
	\frac{\partial c}{\partial t} - D\frac{\partial^{2} c}{\partial x^{2}} = 0
	\label{eq:diffusion}
\end{equation}

Diffusion equation is a partial differential equation (PDE)

We need auxiliary equation conditions:
\begin{equation*}
\begin{aligned}
	\frac{\partial c}{\partial t} = D\frac{\partial^{2} c}{\partial x^{2}},\ \ \  \left. \begin{array}{c}
		a < x < b\\
		t > 0
	\end{array} \right\}
\end{aligned}
\end{equation*}

\begin{itemize}
	\item What was the concentration initially (at the start, $t=0$)
	\item What happens at the endpoints $a$ and $b$.
\end{itemize}

\subsection{Aside}\label{subsec:aside}
Diffusion equations arise in many other settings\\
\subsubsection{Heat transfer}\label{subsubsec:heat-transfer}
Heat energy is measured by temperature $[\theta]$

Let $u(x,t) = $ temperature in the bar at $x$ and time $t$.

Same derivation as before:
\[ \frac{\partial u}{\partial t} = -\frac{\partial J}{\partial x} \]
Now: $J(x,t) = $ \textit{heat} flux

Fourier Law of Heat Conduction:
\begin{equation}
\begin{aligned}
	J &= -D\frac{\partial u}{\partial x}\\
	\frac{\partial u}{\partial t} &= D\frac{\partial^{2}u}{\partial x^{2}}\\
\end{aligned}
	\label{eq:fourier-heat}
\end{equation}
where $D=$ thermal diffusion constant

\subsubsection{Probability}\label{subsubsec:probability}
Let $p(x,t) = $ probability density function of a stochastic process $X(t)$

\begin{equation}
	Pr\left[ a \leq X(t) < b \right] = \int_{a}^{b} p(x,t) dx
	\label{eq:probability-density-function}
\end{equation}

Example: Random wale (discrete) $\rightarrow$ Brownian motion (continuous)

\textbf{Brownian motion's pdf}:
\begin{equation}
	\frac{\partial p}{\partial t} = \frac{1}{2}\frac{\partial^{2} p}{\partial x^{2}},\ \ -\infty < x < \infty, t > 0
	\label{eq:brownian-probability-function}
\end{equation}
where the initial condition $(X(0)=0) \Rightarrow p(x,0) = \delta$ (The dirac delta function)
\[ \delta(x) = \left\{ \begin{array}{cc}
	0 & x\neq 0\\
	\infty & x = 0\\
\end{array} \right. \]
such that
\[ \int_{-\infty}^{\infty} \delta(x)dx = 1 \]


\subsubsection{Biomedical Application}\label{subsubsec:biomedical-application}
Drug patch concentration $=u_{0}$ (fixed)

$u(x,t) = $ concentration at position $x$ (depth) and time $t$ ($x=0$ is the surface of the skin)
\[ \frac{\partial u}{\partial t} = D\frac{\partial^{2}u}{\partial x^{2}},\ \ 0 < x < \infty,\ \ t > 0 \]

Initial condition: $u(x,0) = 0$.

Boundary condition: $u(0, t) = u_{0}$, $u\rightarrow0, x\rightarrow\infty$

\section{Dimensional Analysis}\label{sec:dimensional-analysis}
What is $[D]$?

\begin{equation*}
\begin{aligned}
	\left[ \frac{\partial u}{\partial t} \right] &= \left[ D\frac{\partial^{2}u}{\partial x^{2}} \right]\\
	\left[ \frac{\partial u}{\partial t} \right] &= [D]\left[ \frac{\partial^{2}u}{\partial x^{2}} \right]\\
	ML^{-1}T^{-1} &= [D]ML^{-3}\\
	L^{2}T^{-1} &= [D]\\
\end{aligned}
\end{equation*}

\begin{center}
	\textcolor{blue}{Exact solution of $u$ is hard!\\Try Laplace transforms}
\end{center}

\subsection{Alternate Approach}\label{subsec:alternate-approach}
\begin{enumerate}
	\item Dimensional reduction to understand the form of the solution
	\item Get non-dimensional problem
	\item Use dimensionless variables to convert PDE $\rightarrow$ ODE
\end{enumerate}

\begin{equation*}
\begin{aligned}
	u &= f(x, t, D, u_{0})\\
	[u] &= [x]^{a}[t]^{b}[D]^{c}[u_{0}]^{d}\\
	ML^{-1} &= L^{a}T^{b}\left( L^{2}T^{-1} \right)^{c}\left( ML^{-1} \right)^{d}\\
	M^{1}L^{-1}T^{0} &= L^{a}T^{b}L^{2c}T^{-c}M^{d}L^{-d}\\
	M^{1}L^{-1}T^{0} &= L^{a+2c-d}T^{b-c}M^{d}\\
\end{aligned}
\end{equation*}
\begin{equation*}
\begin{aligned}
	\left[ \begin{array}{rrrr}
		1 & 0 &  2 & -1\\ % L
		0 & 1 & -1 & 0\\ % T
		0 & 0 &  0 & 1\\ % M
	\end{array} \right]
	\left[ \begin{array}{c}
		a\\
		b\\
		c\\
		d\\
	\end{array} \right]
	= \left[ \begin{array}{r}
		-1\\
		0\\
		1\\
	\end{array} \right]
\end{aligned}
\end{equation*}
\begin{equation*}
\begin{aligned}
	a + 2c - d &= -1 \sep b - c &= 0 \sep d &= 1\\
	a &= -2c \sep b &= c \sep d &= 1\\
\end{aligned}
\end{equation*}
\begin{equation*}
\begin{aligned}
	\left[ \begin{array}{c}
		a\\
		b\\
		c\\
		d\\
	\end{array} \right]
	&= \left[ \begin{array}{c}
		-2c\\
		c\\
		c\\
		1\\
	\end{array} \right]\\
	&= \left[ \begin{array}{c}
		0\\
		0\\
		0\\
		1\\
	\end{array} \right]
	+ c\left[ \begin{array}{c}
		-2\\
		1\\
		1\\
		0\\
	\end{array} \right]\\
	&= u_{0}^{1} + \left( t^{1}D^{1}x^{-2} \right)^{c}\\
	&= u_{0} + \left( \frac{tD}{x^{2}} \right)^{c}\\
	u(x,t) &= u_{0}F\left( \frac{tD}{x^{2}} \right)
\end{aligned}
\end{equation*}
Boundary condition makes us multiple the vector within the nullspace by $-\frac{1}{2}$.

\begin{equation*}
\begin{aligned}
	-\frac{1}{2}\left[ \begin{array}{c}
		-2\\
		1\\
		1\\
		0\\
	\end{array} \right]
	&= \left[ \begin{array}{c}
		1\\
		-\frac{1}{2}\\
		-\frac{1}{2}\\
		0\\
	\end{array} \right]\\
\end{aligned}
\end{equation*}
\begin{equation*}
\begin{aligned}
	\left[ \begin{array}{c}
		a\\
		b\\
		c\\
		d\\
	\end{array} \right]
	&= \left[ \begin{array}{c}
		0\\
		0\\
		0\\
		1\\
	\end{array} \right]
	+ c\left[ \begin{array}{c}
		1\\
		-\frac{1}{2}\\
		-\frac{1}{2}\\
		0\\
	\end{array} \right]\\
	&= u_{0}^{1} + \left( t^{-\frac{1}{2}}D^{-\frac{1}{2}}x^{1} \right)^{c}\\
	&= u_{0} + \left( \frac{x}{\sqrt{tD}} \right)^{c}\\
	u(x,t) &= u_{0}F\left( \frac{x}{\sqrt{tD}} \right)\\
	\frac{u(x,t)}{u_{0}} &= F\left( \frac{x}{\sqrt{tD}} \right)\\
\end{aligned}
\end{equation*}

\[ \frac{u}{u_{0}} = F\left( \frac{x}{\sqrt{DT}} \right) \]
is a huge reduction.

Introduce a non-dimensional variable into $u$.
\[ v(x, t) = \frac{u(x,t)}{u_{0}}, \]
$[v] = 1, 0 \leq v < 1$

We can make the similarity variable $\eta = \frac{x}{\sqrt{tD}}$.
$[\eta]=1$

\begin{equation*}
\begin{aligned}
	u(x,t) &= v(\eta)\\
	u_{t} = Du_{xx} &\Rightarrow v_{t} = Dv_{xx}
\end{aligned}
\end{equation*}
The chain rule from Multivariate calculus says that $v(x,t) = v(\eta(x, t))$
\begin{equation*}
\begin{aligned}
	\frac{\partial}{\partial t} = \frac{\partial}{\partial \eta}\frac{\partial \eta}{\partial t} = -\frac{1}{2}\frac{x}{\sqrt{D}}\frac{1}{t^{3/2}}\frac{\partial}{\partial \eta}\\
	\frac{\partial}{\partial x} = \frac{\partial}{\partial \eta}\frac{\partial \eta}{\partial x} = \frac{1}{\sqrt{tD}}\frac{\partial}{\partial n}\\
	\frac{\partial^{2}}{\partial x^{2}} = \frac{1}{Dt}\frac{\partial^{2}}{\partial \eta^{2}} \textcolor{emphorange}{\leftarrow \mbox{ be able to derive!!}} % TODO: Find out what happened to the square root
\end{aligned}
\end{equation*}

Plug into $u_{t} = Du_{xx}$
\begin{equation*}
\begin{aligned}
	-\frac{1}{2}\frac{x}{\sqrt{D}}\frac{1}{t^{3/2}}v' &= D\frac{1}{Dt}v''\\
	-\frac{1}{2}\frac{x}{\sqrt{D}}\frac{1}{t^{3/2}}v' &= \frac{1}{t}v''\\
	-\frac{1}{2}\frac{x}{\sqrt{D}}\frac{1}{t^{1/2}}v' &= v''\\
	-\frac{1}{2}\frac{x}{\sqrt{D}t^{1/2}}v' &= v''\\
	-\frac{1}{2}\frac{x}{\sqrt{Dt}}v' &= v''\\
	-\frac{1}{2}\eta v' &= v''\\
	v'' + \frac{1}{2}\eta v' &= 0\\
\end{aligned}
\end{equation*}

Initial conditional(s) and boundary conditions must be consistent
Boundary conditions:
\begin{equation*}
\begin{aligned}
	\frac{u(0, t)}{u_{0}} = \frac{u_{0}}{u_{0}} \Rightarrow v(0) = 1\\\\
	x = 0 \Rightarrow \eta = \frac{x}{Dt} = 0\\
	u \rightarrow 0 \mbox{ as } x \rightarrow \infty \Rightarrow v \rightarrow 0 \mbox{ if } \eta \rightarrow \infty
\end{aligned}
\end{equation*}
Initial condition:
\begin{equation*}
\begin{aligned}
	u(x, 0) = 0\\
	t = 0 \Rightarrow \eta \rightarrow \infty \Rightarrow v(\infty) = 0
\end{aligned}
\end{equation*}

\vspace{1em}
\begin{center}
	Boundary Value ODE Problem
	\begin{equation*}
	\begin{aligned}
		v'' + \frac{1}{2}\eta v' = 0,\ \ 0 < \eta < \infty\\
		v(0) = 1,\ \ v(\infty) = 0
	\end{aligned}
	\end{equation*}
	2nd order, linear, not constant coefficient homogeneous.
\end{center}

Let $g = v'$
\begin{equation*}
\begin{aligned}
	g' + \frac{1}{2}\eta g &= 0\\
	I &= e^{\int \frac{1}{2}\eta d\eta}\\
	  &= e^{\frac{\eta^{2}}{4}}\\
	e^{\frac{\eta^{2}}{4}}g' + \frac{1}{2}\eta e^{\frac{\eta^{2}}{4}}g &= 0 \\
	\left( e^{\frac{\eta^{2}}{4}}g \right)' &= 0\\
	e^{\frac{\eta^{2}}{4}}g &= C\\
	g &= Ce^{-\frac{\eta^{2}}{4}}\\
	  &= c_{2}e^{-\frac{\eta^{2}}{4}}\\
\end{aligned}
\end{equation*}
\begin{equation*}
\begin{aligned}
	v &= \int g d\eta\\
	  &= c_{1} + c_{2}\int_{0}^{\eta} e^{-\frac{\zeta^{2}}{2}}d\zeta\\ % TODO: Find out how the denomiator changed from 2 to 4
	v(0) = 1 &\Rightarrow c_{1} = 1\\
	v &= 1 + c_{2}\int_{0}^{\eta} e^{-\frac{\zeta^{2}}{2}}d\zeta\\
	v(\infty) &= 0\\
	0 &= 1 + c_{2}\int_{0}^{\infty} e^{-\frac{\zeta^{4}}{2}}d\zeta\\
	0 &= 1 + c_{2}[ \sqrt{\pi} ]\\
	-1 &= c_{2}\sqrt{\pi}\\
	-\frac{1}{\sqrt{\pi}} &= c_{2}\\
\end{aligned}
\end{equation*}
\[	v(\eta) = 1 - \frac{1}{\sqrt{\pi}}\int_{0}^{\eta}e^{\frac{-s^{2}}{4}} ds \]
Change this back to the original variable: $\eta = \frac{x}{\sqrt{DT}}, v = \frac{u}{u_{0}}$
\begin{equation*}
\begin{aligned}
	\frac{u(\eta)}{u_{0}} &= 1 - \frac{1}{\sqrt{\pi}}\int_{0}^{\eta}e^{\frac{-s^{2}}{4}} ds\\
	\frac{u(x, t)}{u_{0}} &= 1 - \frac{1}{\sqrt{\pi}}\int_{0}^{x/\sqrt{Dt}}e^{\frac{-s^{2}}{4}} ds\\
	u(x, t) &= u_{0}\left[ 1 - \frac{1}{\sqrt{\pi}}\int_{0}^{x/\sqrt{Dt}}e^{\frac{-s^{2}}{4}} ds \right]\\
			&= u_{0}\left[ 1 - \mbox{erf}\left( \frac{x}{2\sqrt{Dt}} \right) \right]\\
\end{aligned}
\end{equation*}
where erf is the error function
\begin{equation}
	\mbox{erf}(y) = \frac{2}{\sqrt{\pi}}\int_{0}^{y} e^{-s^{2}}ds
	\label{eq:erf}
\end{equation}
\[ \mbox{erfc} = 1 - \mbox{erf} \]
\begin{equation*}
\begin{aligned}
	u(x, t) &= u_{0}\left[ 1 - \mbox{erf}\left( \frac{x}{2\sqrt{Dt}} \right) \right]\\
			&= u_{0}\mbox{erfc}\left( \frac{x}{2\sqrt{Dt}} \right)\\
\end{aligned}
\end{equation*}

\begin{center}
	\textbf{Method of Similarity Variables}
\end{center}

\subsection{Aside 2 (Amother way of writing \mbox{erfc})}\label{subsec:aside-2}
\begin{equation}
	N(x) = \mbox{ cumulative normal distribution } = \int_{-\infty}^{x}\frac{e^{-\frac{s^{2}}{2}}ds}{\sqrt{2\pi}}
	\label{eq:cumulative-normal-distribution}
\end{equation}
$N(-\infty) = 0, N(\infty) = 1$
\[ N(x) = \frac{1 + \mbox{erf}\left( \frac{x}{\sqrt{2}} \right)}{2} \]

\end{document}