%! Author = Len Washington III
%! Date = 9/04/24

% Preamble
\documentclass[
	date={September 4{,} 2024}
]{math486notes}

% Document
\begin{document}

Let $c(x,t) = $ concentration of a chemical in a region $(x, x+dx)$ at time $t$.

In 3 dimensions: $[c(x,t)] = ML^{-3}$.

In 1 dimension: $[c(x,t)] = ML^{-1}$.

Key: $c(x,t)$ is measurable

\begin{equation*}
\begin{aligned}
	\int_{a}^{b} c(x,t)dx = \mbox{ total concentration of chemical in } (a, b) \mbox{ at time }t
\end{aligned}
\end{equation*}
Check the units
\begin{equation*}
\begin{aligned}
	\left[ \int_{a}^{b} c(x,t)dx \right] = ML^{-1}\times L = M
\end{aligned}
\end{equation*}
How does a chemical move?
\definition{flux $J(x,t)$}{amount of substance that passes through $x$ in the positive direction per unit time.}

The rate of change of the amount is
\[ \frac{d}{dt}\int_{a}^{b} c(x,t)dx = J(a,t) - J(b,t) \]
or
\[ \int_{a}^{b} \left( \frac{\partial c}{\partial t} + \frac{\partial J}{\partial x} \right)dx = 0 \]
This is the \textbf{Conservation of Mass!}
\[ \frac{\partial c}{\partial t} + \frac{\partial J}{\partial x} = 0 \]

What is the flux $J$?

How is it related to $c(x,t)$?

\section{Chemical Diffusion}\label{sec:chemical-diffusion}
\definition{Fick's Law}{chemical moves from regions of higher concentration to lower concentration.}
\[ J \propto -\frac{\partial c}{\partial x} \]
or
\[ J = -D\frac{\partial c}{\partial x} \]
where $D$ is the diffusion constant.

Substitute $J=-D\frac{\partial c}{\partial d}$ in
\[ \frac{\partial c}{\partial t} + \frac{\partial J}{\partial x} = 0 \]

\begin{equation}
	\frac{\partial c}{\partial t} - D\frac{\partial^{2} c}{\partial x^{2}} = 0
	\label{eq:diffusion}
\end{equation}

Diffusion equation is a partial differential equation (PDE)

We need auxiliary equation conditions

\begin{itemize}
	\item What was the concentration initially (at the start, $t=0$)
	\item What happens at the endpoints $a$ and $b$.
\end{itemize}

\subsection{Aside}\label{subsec:aside}
Diffusion equations arise in many other settings\\
\subsubsection{Heat transfer}\label{subsubsec:heat-transfer}
Heat energy is measured by temperature $[\theta]$

Let $u(x,t) = $ temperature in the bat at $x$ and time $t$.

Same derivation as before:
\[ \frac{\partial u}{\partial t} = -\frac{\partial J}{\partial x} \]
Now: $J(x,t) = $ heat flux

Fourier Law of Heat Conduction:
\begin{equation}
\begin{aligned}
	J &= -D\frac{\partial u}{\partial x}\\
	\frac{\partial u}{\partial t} &= D\frac{\partial^{2}u}{\partial x^{2}}\\
\end{aligned}
	\label{eq:fourier-heat}
\end{equation}

\subsubsection{Probability}\label{subsubsec:probability}
Let

\subsubsection{Biomedical Application}\label{subsubsec:biomedical-application}
Drug patch concentration $=u_{0}$ (fixed)

$u(x,t) = $ concentration at position $x$ (depth) and time $t$
\[ \frac{\partial u}{\partial t} = D\frac{\partial^{2}u}{\partial x^{2}},\ \ 0 < x < \infty,\ \ t > 0 \]

Initial condition: $u(x,0) = 0$.

Boundary condition: $u(0, t) = u_{0}$, $u\rightarrow0, x\rightarrow\infty$

\section{Dimensional Analysis}\label{sec:dimensional-analysis}
What is $[D]$?

\begin{equation*}
\begin{aligned}
	\left[ \frac{\partial u}{\partial t} \right] &= \left[ D\frac{\partial^{2}u}{\partial x^{2}} \right]\\
	\left[ \frac{\partial u}{\partial t} \right] &= [D]\left[ \frac{\partial^{2}u}{\partial x^{2}} \right]\\
	ML^{-1}T^{-1} &= [D]ML^{-3}\\
	L^{2}T^{-1} &= [D]\\
\end{aligned}
\end{equation*}

\subsection{Alternate Approach}\label{subsec:alternate-approach}
\begin{enumerate}
	\item Dimensional reduction to understand the form of the solution
	\item Get non-dimensional problem
	\item Use dimensionless variables to convert PDE $\rightarrow$ ODE
\end{enumerate}

\begin{equation*}
\begin{aligned}
	u &= f(x, t, D, u_{0})\\
	[u] &= [x]^{a}[t]^{b}[D]^{c}[u_{0}]^{d}\\
	ML^{-1} &= L^{a}T^{b}\left( L^{2}T^{-1} \right)^{c}\left( ML^{-1} \right)^{d}\\
	M^{1}L^{-1}T^{0} &= L^{a}T^{b}L^{2c}T^{-c}M^{d}L^{-d}\\
	M^{1}L^{-1}T^{0} &= L^{a+2c-d}T^{b-c}M^{d}\\
\end{aligned}
\end{equation*}
\begin{equation*}
\begin{aligned}
	\left[ \begin{array}{rrrr}
		1 & 0 &  2 & -1\\ % L
		0 & 1 & -1 & 0\\ % T
		0 & 0 &  0 & 1\\ % M
	\end{array} \right]
	\left[ \begin{array}{c}
		a\\
		b\\
		c\\
		d\\
	\end{array} \right]
	= \left[ \begin{array}{r}
		-1\\
		0\\
		1\\
	\end{array} \right]
\end{aligned}
\end{equation*}
\begin{equation*}
\begin{aligned}
	a + 2c - d &= -1 \sep b - c &= 0 \sep d &= 1\\
	a &= -2c \sep b &= c \sep d &= 1\\
\end{aligned}
\end{equation*}
\begin{equation*}
\begin{aligned}
	\left[ \begin{array}{c}
		a\\
		b\\
		c\\
		d\\
	\end{array} \right]
	&= \left[ \begin{array}{c}
		-2c\\
		c\\
		c\\
		1\\
	\end{array} \right]\\
	&= \left[ \begin{array}{c}
		0\\
		0\\
		0\\
		1\\
	\end{array} \right]
	+ c\left[ \begin{array}{c}
		-2\\
		1\\
		1\\
		0\\
	\end{array} \right]\\
	&= u_{0}^{1} + \left( t^{1}D^{1}x^{-2} \right)^{c}\\
	&= u_{0} + \left( \frac{tD}{x^{2}} \right)^{c}\\
	u(x,t) &= u_{0}F\left( \frac{tD}{x^{2}} \right)
\end{aligned}
\end{equation*}
Boundary condition makes us multiple the vector within the nullspace by $-\frac{1}{2}$.

\begin{equation*}
\begin{aligned}
	-\frac{1}{2}\left[ \begin{array}{c}
		-2\\
		1\\
		1\\
		0\\
	\end{array} \right]
	&= \left[ \begin{array}{c}
		1\\
		-\frac{1}{2}\\
		-\frac{1}{2}\\
		0\\
	\end{array} \right]\\
\end{aligned}
\end{equation*}
\begin{equation*}
\begin{aligned}
	\left[ \begin{array}{c}
		a\\
		b\\
		c\\
		d\\
	\end{array} \right]
	&= \left[ \begin{array}{c}
		0\\
		0\\
		0\\
		1\\
	\end{array} \right]
	+ c\left[ \begin{array}{c}
		1\\
		-\frac{1}{2}\\
		-\frac{1}{2}\\
		0\\
	\end{array} \right]\\
	&= u_{0}^{1} + \left( t^{-\frac{1}{2}}D^{-\frac{1}{2}}x^{1} \right)^{c}\\
	&= u_{0} + \left( \frac{x}{\sqrt{tD}} \right)^{c}\\
	u(x,t) &= u_{0}F\left( \frac{x}{\sqrt{tD}} \right)\\
	\frac{u(x,t)}{u_{0}} &= F\left( \frac{x}{\sqrt{tD}} \right)\\
\end{aligned}
\end{equation*}

We can make the similarity variable $\eta = \frac{x}{\sqrt{tD}}$.
$[\eta]=1$

\begin{equation*}
\begin{aligned}
	u(x,t) &= v(\eta)
	% TODO: Get the rest of the notes
\end{aligned}
\end{equation*}


\begin{equation}
	N(x) = \mbox{ cumulative normal distribution } = \int_{}^{}
	\label{eq:cumulative-normal-distribution}
\end{equation}

\end{document}