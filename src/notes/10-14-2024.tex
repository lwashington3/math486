%! Author = Len Washington III
%! Date = 10/14/24

% Preamble
\documentclass[
	date={October 14{,} 2024},
	month={10},
	day={14}
]{math486notes}

% Document
\begin{document}

\tableofcontents

\section{Chemical Kinetics}\label{sec:chemical-kinetics}
Chemical reactions, well-stirred (no diffusion)

Reactions occur due ot collision between the molecules
\[ A + B \rightarrow^{k} C \]
where $k$ is the rate.

How does the reaction proceed -- \emph{to track!}
Use concentration of the chemicals

Notation: $A(t)$ = concentration of $A$

$[A] = \frac{\mbox{moles}}{\mbox{concentration}^{3}}$

\begin{equation}
	\frac{dA}{dt} = \frac{dB}{dt} = -\frac{dC}{dt}
	\label{eq:reaction-rate}
\end{equation}

\section{Simple Reactions}\label{sec:simple-reactions}
\subsection{1st order: $A \rightarrow^{k} B$ (radioactive decay)}\label{subsec:1st-order-reaction}
\begin{equation*}
\begin{aligned}
	\frac{dA}{dt} &\propto A\\
	\frac{dA}{dt} &= kA^{1}\\
	A &= ce^{-kt}\\
	\ln A &= -kt + c
\end{aligned}
\end{equation*}

\subsection{2nd order: $A + A \rightarrow^{k} B$}\label{subsec:2nd-order-reaction}
\begin{equation*}
\begin{aligned}
	2A &\rightarrow^{k} B\\
	\frac{dA}{dt} &= -2kA^{2}\\
\end{aligned}
\end{equation*}

\begin{equation*}
\begin{aligned}
	4\mbox{Li} + \mbox{O}_{2} \rightarrow 2\mbox{Li}_{2}\mbox{O}
\end{aligned}
\end{equation*}
The stoicheiometric coefficients are 4, 1, and 2.

Goal: Track concentrations of chemicals in time (similar to drug care)

\section{Law of Mass Action}\label{sec:law-of-mass-action}
\begin{enumerate}[label=\arabic*)]
	\item The rate $r$ of the reaction is proportional to the product of the reactant concentrations, with each concentration raised to the power equal to is respective stoichiometric coefficient.
	\item The rate of change of the concentration of each species in the reaction is the product of its stoichiometric coefficients with the rate of the reaction adjusted for sign ($-$ for reactant, $+$ for product).
	\item For a system of reactions,
	\begin{equation}
		\alpha A + \beta B \rightarrow^{k} \gamma C + \delta D\\
		\label{eq:law-of-mass-action}
	\end{equation}
\end{enumerate}
Using~\eqref{eq:law-of-mass-action}:
\begin{equation*}
\begin{aligned}
	\frac{dA}{dt} &= -\alpha r\\
				  &= -\alpha k A^{\alpha} B^{\beta}\\
	\frac{dB}{dt} &= -\beta k A^{\alpha} B^{\beta}\\
	\frac{dC}{dt} &= \gamma k A^{\alpha} B^{\beta}\\
	\frac{dD}{dt} &= \delta k A^{\alpha} B^{\beta}\\
\end{aligned}
\end{equation*}

\begin{equation*}
\begin{aligned}
	A + 2x \rightarrow^{k} P \Rightarrow r = kA^{1}x^{2}
\end{aligned}
\end{equation*}
This is highly unlikely given that the reaction would need 2 $x$'s to come into contact with an $A$ at the same time.

\begin{center}
	Reaction may proceed in elementary steps!
\end{center}

True mechanism
\begin{equation*}
\begin{aligned}
	\left\{ \begin{array}{l}
		A + x \rightarrow^{k_{1}} Z\\
		Z + x \rightarrow^{k_{2}} p\\
	\end{array} \right.
\end{aligned}
\end{equation*}
where $z$ is an intermediate complex!
This mechanism only involves binary collisions.

\begin{equation*}
\begin{aligned}
	\mbox{Law of mass Action } \Rightarrow
	\left\{ \begin{array}{l}
		\frac{dA}{dt} = -k_{1}Ax\\
		\frac{dx}{dt} = -k_{1}Ax- k_{2}zx \\
		\frac{dz}{dt} = k_{1}Ax - k_{2}xz\\
		\frac{dP}{dt} = k_{2}xz\\
	\end{array} \right.
\end{aligned}
\end{equation*}

4 ODE's -- nonlinear (hard)

Initial conditions: $A(0)=A_{0}$, $x(0) = x_{0}$, $z(0) = 0$, $P(0) = 0$.

How do we expect the concentration to change in time.

How to analyze a nonlinear, system of ODE's.

Strategy: Reduce the number of equations: \textbf{compression}.

\end{document}