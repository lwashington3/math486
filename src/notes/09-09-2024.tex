%! Author = Len Washington III
%! Date = 9/09/24

% Preamble
\documentclass[
	date={September 9{,} 2024}
]{math486notes}

% Document
\begin{document}

\section{Pharmacology -- how drugs behave in the body}\label{sec:pharmacology}
\subsection{Phramacokinetics (PK)}\label{subsec:phramacokinetics-(pk)}
Movement of drugs in the body

\subsection{Pharmacodynamics (PD)}\label{subsec:pharmacodynamics-(pd)}
Body's response to drugs\\

We will focus more on PK (movement of drugs)
\begin{itemize}
	\item absorption
	\item distribution
	\item elimination or clearance
\end{itemize}

FDA - Federal Drug Administration
Questions:
\begin{itemize}
	\item Dosage -- what is appropriate
	\item Toxicity
	\item Side effects!
\end{itemize}

The upper level is the minimum toxic level (MToL) and the lower level is the minimum effective therapeutic level (MTL).
The concentration of the drug within the body must stay within the therapeutic region window $(MTL < conc < MToL)$.

\section{Modeling Tool: Compartment Analysis}\label{sec:modeling-tool:-compartment-analysis}
Let $x(t)$ = the amount of drug in blood at time $t$.

Multiple compartments (kidneys, liver, etc).

\begin{center}
	Question: How is drug removed from a compartment?
\end{center}
Elimination, clearance or removal process.

\subsection{Simple Assumption}\label{subsec:simple-assumption}
First Order Chemical Reaction!
($A^{k} \rightarrow $ products)

\textbf{Chemical Kinetics}
$x(t) = $ amount
\[ \frac{dx}{dt} = -kx,\ \ x(0) = x_{0} \]

\begin{example}[IV drip delivery system - one compartment model]
	IV infusion $\rightarrow$ blood $\rightarrow$

	$I$ = infusion rate, $[I] = MT^{-1}$

	ODE: ($x(0) = 0$)
	\begin{equation*}
	\begin{aligned}
		\frac{dx}{dt} &= \mbox{input} + \mbox{output (removal)}\\
		\frac{dx}{dt} &= I - kx\\
	\end{aligned}
	\end{equation*}
	\begin{equation*}
	\begin{aligned}
		\left[ \frac{dx}{dt} \right] &= [I] - [kx]\\
		MT^{-1} &= MT^{-1} - [k]M\\
		MT^{-1} &= [k]M\\
		T^{-1} &= [k]\\
		[k] &= T^{-1}\\
	\end{aligned}
	\end{equation*}
	\begin{equation*}
	\begin{aligned}
		\frac{dx}{dt} &= I - kx\\
		\frac{dx}{dt} + kx &= I\\
		I.F. &= e^{\int k dt}\\
			 &= e^{k\int dt}
			 &= e^{kt}\\
		e^{kt}x' + ke^{kt}x &= Ie^{kt}\\
		\frac{d}{dt}\left( e^{kt}x \right) &= Ie^{kt}\\
		e^{kt}x &= \int Ie^{kt}\\
		e^{kt}x &= \frac{I}{k}e^{kt} + C\\
		x &= \frac{I}{k} + Ce^{-kt}\\
	\end{aligned}
	\end{equation*}
	\begin{equation*}
	\begin{aligned}
		0 &= \frac{I}{k} + Ce^{-k(0)}\\
		0 &= \frac{I}{k} + Ce^{0}\\
		0 &= \frac{I}{k} + C(1)\\
		-C &= \frac{I}{k}\\
		C &= -\frac{I}{k}\\
	\end{aligned}
	\end{equation*}
	\begin{equation*}
	\begin{aligned}
		x &= \frac{I}{k} - \frac{I}{k}e^{-kt}\\
		x &= \frac{I}{k} \left( 1 - e^{-kt} \right)\\
	\end{aligned}
	\end{equation*}
\end{example}

\section{Pk Metrics}\label{sec:pk-metrics}
$x(t) = $ amount in plasma
$c(t) = \frac{x(t)}{V_{blood}}$ where the average amount of blood ($V_{blood}$) in a person is $\approx 5L$. This is \emph{incorrect} because it does not match the data.

\begin{equation}
\begin{aligned}
	V_{D} &= \mbox{ apparent volume of distribution.}\\ % TODO: Check this
	V_{D} &= \frac{\mbox{amount of drug entering the body}}{\mbox{initial concentration in the plasma}}\\
		  &= \frac{x(0)}{c(0)}
\end{aligned}
	\label{eq:}
\end{equation}
This data is based off a 70 kilogram man.

\begin{table}[H]
	\centering
	\caption{}
	\label{tab:}
	\begin{tabular}{ll}
		\textbf{Drug} & $\mathbf{V_{D}}$\\
		warfarin & 8L\\
		theophline & 30L\\
		chloroquine & 15000L -- absorbed in fat
	\end{tabular}
\end{table}

\section{Area Under Curve (AUC)}\label{sec:area-under-curve-(auc)}
Measures the total exposure to the drug
\[ AUC = \frac{\mbox{total amount}}{k} \]


\end{document}