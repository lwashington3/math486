%! Author = Len Washington III
%! Date = 8/28/24

% Preamble
\documentclass[
	date={August 28{,} 2024}
]{math486notes}

% Document
\begin{document}

$R_{e}$ small -- laminar flow.\\
$R_{e}$ large -- turbulent flow.

\section{Why is $R_{e}$ useful}\label{sec:why-is-reynolds-useful}
Prototyping ($p$) using a model ($m$)
\begin{enumerate}
	\item Geometric similarity
	\item Kinematic or dynamic similarity
\end{enumerate}

Assuming
\begin{enumerate}
	\item materials are the same
	\item both in the same medium
\end{enumerate}
then
\begin{equation*}
\begin{aligned}
	\rho_{p} &= \rho_{m}\\
	\mu_{p} &= \mu_{m}\\
	R_{e_{p}} &= R_{e_{m}}\\
	\frac{R_{m}v_{m}\rho_{m}}{\mu_{m}} &= \frac{R_{p}v_{p}\rho_{p}}{\mu_{m}}\\
	v_{m} &= \frac{R_{p}v_{p}\rho_{p}\mu_{m}}{\mu_{m}R_{m}\rho_{m}}\\
		  &= \frac{\rho_{p}}{\rho_{m}}\times\frac{\mu_{p}}{\mu_{m}}\times\frac{R_{p}v_{p}}{R_{m}}\\
		  &= \frac{R_{p}}{R_{m}}v_{p}\\
\end{aligned}
\end{equation*}

\setcounter{chapter}{1}
\chapter{Scaling and Nondimensionality}\label{ch:scaling-and-nondimensionality}

\[ \frac{dP}{dt} = rP\left( 1 - \frac{P}{k} \right), \ \ p(0)=p_{0} \]

\begin{equation*}
\begin{aligned}
	p &= f(t, r, k, p_{0})
\end{aligned}
\end{equation*}

\begin{itemize}
	\item $[p] = \#$
	\item $[t] = T$
	\item $[p_{0}] = \#$
\end{itemize}

\[ \frac{dP}{dt} = rP - \frac{rP^{2}}{k} \]
\begin{itemize}
	\item $\left[ \frac{dP}{dt} = \frac{\#}{T} \right]$
	\item $[rP] = [r][P] = $
\end{itemize}

\begin{equation*}
\begin{aligned}
	p &= f(t, r, k, p_{0})\\
	[p] &= [t]^{a}[r]^{b}[k]^{c}[p_{0}]^{d}\\
	T^{0}\# &= T^{a}(T^{-1})^{b}\#^{c}\#^{d}\\
	T^{0}\# &= T^{a}T^{-b}\#^{c}\#^{d}\\
	T^{0}\# &= T^{a-b}\#^{c+d}\\
\end{aligned}
\end{equation*}

\chapter{Chemical Diffusion}\label{ch:chemical-diffusion}
\begin{center}
	Molecules or particles in a fluid
\end{center}

We do not want to track number -- to large!
Measure concentration instead of number.\\

3 Dimensional $[concentration] = \frac{M}{Volume} = \frac{M}{L^{3}}$

Flux $J(x,t)$ = the amount of substance that passes through $x$ in the positive direction per unit time. (If flow goes to the left: $J < 0$)\\

rate of change of amount:
\begin{equation}
	\frac{d}{dt}\int_{a}^{b} c(x,t)dx = J(a,t) - J(b,t)
	\label{eq:flux}
\end{equation}

\begin{equation}
	\int_{a}^{b} \frac{\partial c}{\partial t} dx = -\int_{a}^{b} \frac{\partial J}{\partial x} dx
	\label{eq:what-the-flux}
\end{equation}

\begin{equation}
	\int_{a}^{b} \left( \frac{\partial c}{\partial t} + \frac{\partial J}{\partial x}\right) dx = 0
	\label{eq:flux2}
\end{equation}

What is the flux -- $J$?
\definition{Chemical diffusion}{how does the flow depend on the concentration?}

Chemical moves from region of high concentration to low concentration.

\begin{equation}
	J \propto -\frac{\partial c}{\partial x}
	\label{eq:Ficks-Law}
\end{equation}

\end{document}