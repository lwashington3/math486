%! Author = Len Washington III
%! Date = 10/30/24

% Preamble
\documentclass[
	date={October 30{,} 2024},
	month={10},
	day={30}
]{math486notes}

% Document
\begin{document}

\tableofcontents

\begin{equation}
	\frac{dC}{dt} = -\frac{V_{max}}{k_{m} + C}C,\ \ C(0^{+}) = C_{0}
	\label{eq:star}
\end{equation}

Constructed an implicit solution to~\eqref{eq:star}

\begin{equation}
	\frac{1}{V_{max}}\left[ (C_{0} - C) + k_{m} \ln\frac{C_{0}}{C} \right]
	\label{eq:diamond}
\end{equation}
\begin{equation*}
\begin{aligned}
	F(C) = t
\end{aligned}
\end{equation*}

Given a time $t$ -- solve~\eqref{eq:diamond} for $C$ numerically (Newton's Method).

BUT -- we would need to know $V_{max}$ and $k_{m}$.

However, we were able to compute the half-life from~\eqref{eq:diamond} -- very useful!

\[ t_{\frac{1}{2}} = \frac{k_{m}}{V_{max}} \ln 2 + \frac{C_{0}}{2V_{max}} \]
where $\left[ \frac{V_{max}}{k_{m}} \right] = \frac{1}{T}$

If we could find $F^{-1}$ then $F^{-1}(F(C)) = F^{-1}(t) \Rightarrow C = F^{-1}(t)$.

\begin{equation}
	\begin{aligned}
		F(c) &= \frac{1}{V_{max}} \left[ (C_{0} - C) + k_{m} \ln \frac{C_{0}}C\right]\\
		&= t
	\end{aligned}
	\label{eq:diamond-2}
\end{equation}

Get a better form of~\eqref{eq:diamond-2}.

\begin{equation*}
\begin{aligned}
	\left[ (C_{0} - C) + k_{m} \ln \frac{C_{0}}{C}\right] &= V_{max}t\\
	\frac{C_{0} - C}{k_{m}} + \ln \frac{C_{0}}{C} &= -\frac{V_{max}}{k_{M}}t\\
	e^{\frac{C_{0} - C}{k_{m}} + \ln \frac{C_{0}}{C}} &= e^{-\frac{V_{max}}{k_{M}}t}\\
	e^{\frac{C_{0} - C}{k_{m}}}e^{\ln \frac{C_{0}}{C}} &= e^{-\frac{V_{max}}{k_{M}}t}\\
	e^{\frac{C_{0} - C}{k_{m}}} \frac{C_{0}}{C} &= e^{-\frac{V_{max}}{k_{M}}t}\\
	e^{\frac{C_{0}}{k_{m}}} e^{-\frac{C}{k_{m}}} \frac{C_{0}}{C} &= e^{-\frac{V_{max}}{k_{M}}t}\\
	e^{-\frac{C}{k_{m}}} \frac{1}{C} &= \frac{1}{C_{0}}e^{-\frac{V_{max}}{k_{M}}t}e^{\frac{-C_{0}}{k_{m}}}\\
\end{aligned}
\end{equation*}

Does $G = xe^{x}$ have an inverse $G^{-1}$?

\begin{equation*}
\begin{aligned}
	G(x) &= xe^{x}\\
	G(0) &= 0\\
	G'(x) &= e^{x} + xe^{x}\\
	&= 0\\
	G''(x) &= 2e^{x} + xe^{x}\\
	G''(0) &= 2\\
	>& 0 \mbox{  (concave up)}
\end{aligned}
\end{equation*}
\begin{equation*}
\begin{aligned}
	G'(-1) &= -\frac{1}{e}\\
	\lim_{x\rightarrow -\infty} xe^{x} &= 0
\end{aligned}
\end{equation*}

On $-1 < x < \infty$, $G(x)$ is 1 to 1 $=\Rightarrow G^{-1}$ exists on $-\infty < x < -1$.

Need a name for $G^{-1}$\\.
Call $G^{-1}(x) =$ Lambert $W(x)$.

\begin{equation*}
\begin{aligned}
	G(G^{-1}(x)) &= G(Lambert\ W(x))\\ &= x
\end{aligned}
\end{equation*}
\begin{equation*}
\begin{aligned}
	G(G^{-1}) &= W(x)e^{W(x)}\\
	G(x) &= g(t)\\
	&= xe^{x}\\
	x &= G^{-1}(G)\\
	&= G^{-1}(g(t))
\end{aligned}
\end{equation*}

becomes $x = W\left( x_{0}e^{x_{0}}e^{-\frac{V_{max}}{k_{m}}}  \right)$ % TODO: Get the rest of this

\begin{equation}
	C(t) = k_{m}W\left( \frac{C_{0}}{k_{M}}e^{\frac{(C_{0} - V_{max})t}{k_{M}}} \right)
	\label{eq:c-lambert}
\end{equation}

\section{Periodic Dosing with non-linear clearance}\label{sec:periodic-dosing-with-non-linear-clearance}
Key: Each does of size $C_{0}$ is given at intervals $\tau$.
Decay rate between doses depends on $C(t)$.
This is different than the previous periodic dosing with 1st order clearance.

Iterate:
\begin{equation*}
\begin{aligned}
	v_{0} &= c_{0}\\
	u_{1} &= k_{m}LambertW\left( \frac{C_{0}}{k_{M}}e^{\frac{C_{0} - V_{max}t}{k_{M}}} \right)\\
	v_{1} &= u_{1} + c_{0}\\
	u_{2} &= k_{m}LambertW\left( \frac{u_{1} +  C_{0}}{k_{M}}e^{\frac{u_{1} + C_{0} - V_{max}t}{k_{M}}} \right)\\
\end{aligned}
\end{equation*}

Do the following limits exist?
\begin{equation*}
\begin{aligned}
	v_{n} \rightarrow v_{\infty}\\
	u_{n} \rightarrow u_{\infty}\\
\end{aligned}
\end{equation*}
Are the limits in the therapeutic window?
$MTL < u_{\infty} < v_{\infty} < MToL$

\section{Cleanup topics}\label{sec:cleanup-topics}
Enzyme with 2 sites.

We expect that the reaction would be
\begin{equation}
	S + E \stackrel[k_{-1}]{k_{1}}{\rightleftarrows} C_{1} + S \stackrel[k_{-2}]{k_{2}}{\rightleftarrows} C_{2} \rightarrow P + E
	\label{eq:mm-2-site}
\end{equation}

Non-realistic reaction for simplicity!
\[ 2S + E \stackrel[k_{-1}]{k_{1}}{\rightleftarrows} C_{2} \stackrel{k_{p}}{\rightarrow} E + 2P \]

\section{Epidemiology -- model of infectious diseases}\label{sec:epidemiology----model-of-infectious-diseases}
Examples:
\begin{itemize}
	\item AIDS -- HIV
	\item H1N1 flu -- Spanish flu (1918-1920)
	\item Malaria
	\item Zika
	\item Hep-C
	\item SARS
	\item MERS
	\item COVID-19
\end{itemize}

Who models:
\begin{itemize}
	\item CDC
	\item WHO
	\item John Hopkins University - COVID data site
\end{itemize}

\subsection{Types of Models}\label{subsec:types-of-models}
\begin{enumerate}
	\item Single - individual
	\begin{itemize}
		\item Viral load $\approx10^{12}$
		\item Effectiveness of specific treatment
	\end{itemize}
	\item Spread in an entire population
	\begin{itemize}
		\item spread of epidemic
	\end{itemize}
	\item Agent based models - flow single individuals in (2) -- Zika in Miami
	\begin{itemize}
		\item Spread in localized area
	\end{itemize}
\end{enumerate}

\section{Simple Epidemic Model}\label{sec:simple-epidemic-model}
Population of fixed size $N$.

Subdivide population into 2 groups.

\begin{enumerate}
	\item Susceptible $S(t)$.
	\begin{itemize}
		\item Health but can contract the disease.
	\end{itemize}
	\item Infectives -- infected individual $I(t)$
	\begin{itemize}
		\item Have disease and can spread it on contact
	\end{itemize}
\end{enumerate}

Assumption: Well-mixed population.

Any individual is equally likely to come in contact with any other individuals $\leftarrow$ cannot identify individuals with disease.

\section{SIS model - compartment model}\label{sec:sis-model---compartment-model}

Use modified ``Law of Mass Action''.
Contact rate $\propto$ $SI$.
$\beta$ = recovery rate, $\mu$ = recovery rate per individual
$E[time\ sick] = \frac{1}{\mu}$.

\begin{equation}
	\begin{aligned}
		\frac{dS}{dt} &= -\beta SI + \mu I \sep S(0) = N - 1\\
		\frac{dI}{dt} &= \beta SI - \mu I \sep I(0) = 1
	\end{aligned}
	\label{eq:sis-model-law-of-mass-action}
\end{equation}

Look for conservation laws:
\begin{equation*}
\begin{aligned}
	\frac{dS}{dt} + \frac{dI}{dt} &= 0\\
	\frac{d}{dt} (S + I) &= 0\\
	S(t) + I(t) &= S(0) + I(0)\\
	I(t) &= N - 1 + 1 - S(t)\\
	I(t) &= N - S(t)\\
\end{aligned}
\end{equation*}

\end{document}