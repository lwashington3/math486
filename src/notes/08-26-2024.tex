%! Author = Len Washington III
%! Date = 8/26/24

% Preamble
\documentclass[
	date={August 26{,} 2024}
]{math486notes}

% Document
\begin{document}

\begin{example}[Pendulum]
	$\phi(t)$ = angular displacement.\\
	$\phi(0)$ = $\theta$ = initial angular displacement.\\
	Let $t_{p} = $ period of the pendulum (time for 1 complete cycle)

	\textbf{What factors determine $\mathbf{t_{p}}$} ($[t_{p}] = T$)?
	\begin{enumerate}
		\item $g$
		\begin{itemize}
			\item $[g] = LT^{-2}$
		\end{itemize}
		\item $m$ -- mass
		\begin{itemize}
			\item $[m] = M$
		\end{itemize}
		\item $r$ -- length
		\begin{itemize}
			\item $[r] = L$
		\end{itemize}
		\item $\theta$ = initial displacement
		\begin{itemize}
			\item $[\theta] = 1$
		\end{itemize}
		\item air resistance (drag)
		\item friction in hinge
	\end{enumerate}

	Assume $t_{p} = f(m, g, r, \theta)$
	Dimension reduction:
	\begin{equation*}
	\begin{aligned}
		[t_{p}] &= [m^{a}g^{b}r^{c}\theta^{d}]\\
		[t_{p}] &= [m]^{a}[g]^{b}[r]^{c}[\theta]^{d}\\
		M^{0}L^{0}T^{1} &= M^{a}\left( LT^{2} \right)^{b} L^{c} \left( M^{0}L^{0}T^{0} \right)^{d}\\
						&= M^{a}L^{b}T^{2b}L^{c} M^{0d}L^{0d}T^{0d}\\
						&= M^{a}L^{b}T^{2b}L^{c} (1)\\
						&= M^{a}L^{b+c}T^{2b}\\
	\end{aligned}
	\end{equation*}
	\begin{equation*}
	\begin{aligned}
		\left[ \begin{array}{cccc}
			 1 & 0 & 0 & 0\\
			 0 & 1 & 1 & 0\\
			0 & -2 & 0 & 0\\
		\end{array} \right]\left[ \begin{array}{c}
			a\\
			b\\
			c\\
			d\\
		\end{array} \right] = \left[ \begin{array}{c}
			0\\
			0\\
			1\\
		\end{array} \right]
	\end{aligned}
	\end{equation*}
	\begin{equation*}
	\begin{aligned}
		\left[ \begin{array}{cccc|c}
			1 &  0 & 0 & 0 & 0\\
			0 &  1 & 1 & 0 & 0\\
			0 & -2 & 0 & 0 & 1\\
		\end{array} \right]
		&= \left( r_{2} \gets 2r_{2} + r_{3} \right)\left[ \begin{array}{cccc|c}
			1 &  0 & 0 & 0 & 0\\
			0 &  0 & 2 & 0 & 1\\
			0 & -2 & 0 & 0 & 1\\
		\end{array} \right]\\
		&= \left( r_{2} \leftrightarrow r_{3} \right)\left[ \begin{array}{cccc|c}
			1 &  0 & 0 & 0 & 0\\
			0 & -2 & 0 & 0 & 1\\
			0 &  0 & 2 & 0 & 1\\
		\end{array} \right]\\
		&= \left( r_{2} \gets -\frac{r_{2}}{2} \right)\left[ \begin{array}{cccc|c}
			1 & 0 & 0 & 0 & 0\\
			0 & 1 & 0 & 0 & -\frac{1}{2}\\
			0 & 0 & 2 & 0 & 1\\
		\end{array} \right]\\
		&= \left( r_{3} \gets \frac{r_{3}}{2} \right)\left[ \begin{array}{cccc|c}
			1 & 0 & 0 & 0 & 0\\
			0 & 1 & 0 & 0 & -\frac{1}{2}\\
			0 & 0 & 1 & 0 & \frac{1}{2}\\
		\end{array} \right]\\
	\end{aligned}
	\end{equation*}
	\begin{equation*}
	\begin{aligned}
		a &= 0 \sep b &= -\frac{1}{2} \sep c &= \frac{1}{2} \sep d &= \mbox{arbitrary constant}\\
	\end{aligned}
	\end{equation*}
	\begin{equation*}
	\begin{aligned}
		\left[ \begin{array}{c}
			a\\
			b\\
			c\\
			d\\
		\end{array} \right] &= \left[ \begin{array}{c}
			0\\
			-\frac{1}{2}\\
			\frac{1}{2}\\
			d\\
		\end{array} \right]\\ &= \left[ \begin{array}{c}
			0\\
			-\frac{1}{2}\\
			\frac{1}{2}\\
			0\\
		\end{array} \right]
		+& d\left[ \begin{array}{c}
			0\\
			0\\
			0\\
			1\\
		\end{array} \right]\\
		&= g^{-\frac{1}{2}}r^{\frac{1}{2}} +& \theta^{1}\\
		&= \sqrt{g^{-1}r} + \theta\\
		&= \sqrt{\frac{r}{g}} + \theta\\
	\end{aligned}
	\end{equation*}
	Note the dimensions:
	\begin{equation*}
	\begin{aligned}
		\left[ \sqrt{\frac{r}{g}} \right] &= \sqrt{\frac{[r]}{[g]}}\\
										  &= \sqrt{\frac{L}{LT^{-2}}}\\
										  &= \sqrt{\frac{1}{T^{-2}}}\\
										  &= \sqrt{T^{2}}\\
										  &= T\\
		[\theta] &= 1\\
		\mbox{nullspace} = \left\{ \left[ \begin{array}{c}
			0\\
			0\\
			0\\
			1\\
		\end{array} \right] \right\} &\Rightarrow \mbox{non-dimensional quantity!}
	\end{aligned}
	\end{equation*}
	\begin{equation*}
	\begin{aligned}
		t_{p} &= \alpha\sqrt{\frac{r}{g}}\theta^{d}
	\end{aligned}
	\end{equation*}
	where $\alpha$ is a non-dimensional constant ($[\alpha] = 1$)
	\[ t_{p} = \sqrt{\frac{r}{g}}\left[ \alpha, d_{1}\theta^{d} + d_{2}\theta^{d} + d_{3}\theta^{d} + \dots \right] \]
	where all the terms within the brackets can be defined as $h(\theta)$ (and the $\alpha$'s are hidden in $h(\theta)$ as well).
	So:
	\[ t_{p} = \sqrt{\frac{r}{g}}h(\theta) \]
	where $h(\theta)$ can be found using data.
	The completely dimensionless form would be:
	\[ \frac{t_{p}}{\sqrt{\frac{r}{g}}} = h(\theta) \]
\end{example}

\begin{example}[Pendulum ODE]
	Derivation of ODE describing motion.\\

	Newton's 2nd Law - rotational form
	\begin{description}
		\item[$\tau$] = torque (force)
		\begin{itemize}
			\item $[\tau] = \frac{ML}{T^{2}}$
		\end{itemize}
		\item[$I$] = moment of inertia (like mass; measures how difficult it is to change the rotational force)
		\item[$\alpha$] = angular acceleration (a)
		\begin{itemize}
			\item $\frac{1}{T^{2}}$
		\end{itemize}
	\end{description}
	\[ \alpha = \frac{\tau}{I} \mbox{  or  } \tau = I\alpha \]
	What is the dimension for $I$?
	\begin{equation*}
	\begin{aligned}
		[\tau] &= [I\alpha]\\
			   &= [I][\alpha]\\
		\frac{ML^{2}}{T^{2}} &= [I]\frac{1}{T^{2}}\\
		ML^{2} &= [I]
	\end{aligned}
	\end{equation*}
	In physics, $I=$ mass $\times$ square of the distance from axis $=mr^{2}$

	\begin{equation*}
	\begin{aligned}
		\tau &= I\frac{d^{2}\psi}{dt^{2}}\\
			 &= \vec{r}\times\vec{f}\\
			 &= -mgr\sin(\psi)\\
		I\frac{d^{2}\psi}{dt^{2}} &= -mgr\sin(\psi)\\
		mr^{2}\frac{d^{2}\psi}{dt^{2}} &= -mgr\sin(\psi)\\
		r\frac{d^{2}\psi}{dt^{2}} &= -g\sin(\psi)\\
		r\frac{d^{2}\psi}{dt^{2}} + g\sin(\psi) &= 0\\
		\frac{d^{2}\psi}{dt^{2}} + \frac{g}{r}\sin(\psi) &= 0\\
	\end{aligned}
	\end{equation*}
	Assuming that $\theta$ is small, $\sin(\psi)\approx\psi$.
	\[ \frac{d^{2}\psi}{dt^{2}} + \frac{g}{r}\psi = 0,\ \ \ \psi(0) = \theta,\ \ \ \psi'(0) = 0 \]
	Since this is a 2nd order, homogenous function, we can assume $\psi=e^{\lambda t}$
	\begin{equation*}
	\begin{aligned}
		0 &= \lambda^{2}e^{\lambda t} + \frac{g}{r}e^{\lambda t}\\
		  &= e^{\lambda t}\left( \lambda^{2} + \frac{g}{r} \right)\\
		  &= \lambda^{2} + \frac{g}{r}\\
		-\lambda^{2} &= \frac{g}{r}\\
		\lambda^{2} &= -\frac{g}{r}\\
		\lambda &= \pm\sqrt{-\frac{g}{r}}\\
				&= \pm\sqrt{\frac{g}{r}}\times\sqrt{-1}\\
				&= \pm\sqrt{\frac{g}{r}}i\\
	\end{aligned}
	\end{equation*}
	\[ \mbox{The fundamental set is } \left\{ e^{i\sqrt{\frac{g}{r}}t}, e^{-i\sqrt{\frac{g}{r}}t} \right\} \]
	Their real form (from ODE course) is
	\begin{equation}
		\psi = c_{1}\cos\left( \sqrt{\frac{g}{r}}t \right) + c_{2}\sin\left( \sqrt{\frac{g}{r}}t \right)
		\label{eq:real}
	\end{equation}
	$c_{1}$ and $c_{2}$ can be found using the initial conditions $\psi(0) = \theta$ and $\psi'(0) = 0$.
	\begin{equation*}
	\begin{aligned}
		\psi(0) &= c_{1}\cos\left( \sqrt{\frac{g}{r}}(0) \right) + c_{2}\sin\left( \sqrt{\frac{g}{r}}(0) \right)\\
		\theta &= c_{1}\cos\left( 0 \right) + c_{2}\sin\left( 0 \right)\\
		\theta &= c_{1}(1) + c_{2}(0)\\
		\theta &= c_{1}\\
		c_{1} &= \theta\\
	\end{aligned}
	\end{equation*}
	\begin{equation*}
	\begin{aligned}
		\psi'(t) &= -\sqrt{\frac{g}{r}}c_{1}\sin\left( \sqrt{\frac{g}{r}}t \right) + \sqrt{\frac{g}{r}}c_{2}\cos\left( \sqrt{\frac{g}{r}}t \right)\\
		\psi'(0) &= -\sqrt{\frac{g}{r}}c_{1}\sin\left( \sqrt{\frac{g}{r}}(0) \right) + \sqrt{\frac{g}{r}}c_{2}\cos\left( \sqrt{\frac{g}{r}}(0) \right)\\
		0 &= -\sqrt{\frac{g}{r}}c_{1}\sin(0) + \sqrt{\frac{g}{r}}c_{2}\cos(0)\\
		0 &= -\sqrt{\frac{g}{r}}c_{1}(0) + \sqrt{\frac{g}{r}}c_{2}(1)\\
		0 &= \sqrt{\frac{g}{r}}c_{2}\\
		0 &= c_{2}\\
		c_{2} &= 0\\
	\end{aligned}
	\end{equation*}
	\begin{equation*}
	\begin{aligned}
		\psi &= c_{1}\cos\left( \sqrt{\frac{g}{r}}t \right) + c_{2}\sin\left( \sqrt{\frac{g}{r}}t \right)\\
		\psi &= \theta\cos\left( \sqrt{\frac{g}{r}}t \right) + 0\sin\left( \sqrt{\frac{g}{r}}t \right)\\
		\psi &= \theta\cos\left( \sqrt{\frac{g}{r}}t \right)
	\end{aligned}
	\end{equation*} % TODO: Find out where 2\pi comes from
\end{example}

\section{Buckingham's Pi Theorem}\label{sec:buckingham's-pi-theorem}
Let $q=$ physical quantity (measurable).
$q$ depends on $n$ parameters and variables $p_{1},\dots,p_{n}$
\[ q = f(p_{1},p_{2},\dots,p_{n}) \]

Assume fundamental dimensions $h, T, M$.
\begin{equation*}
\begin{aligned}
	[q] &= L^{l_{0}}T^{t_{0}}M^{m_{0}}\\
	[p_{i}] &= L^{l_{i}}T^{t_{i}}M^{m_{i}}\\
	[q] &= [p_{1}]^{a_{1}}[p_{2}]^{a_{2}}\dots[p_{n}]^{a_{n}}
\end{aligned}
\end{equation*}
Objective: find $a_{j}$ where $j=1,2,\dots,n$

\begin{equation*}
\begin{aligned}
	L^{l_{0}}T^{t_{0}}M^{m_{0}} &= \left( L^{l_{1}}T^{t_{1}}M^{m_{1}} \right)^{a_{1}}\times\left( L^{l_{2}}T^{t_{2}}M^{m_{2}} \right)^{a_{2}}\times\dots\times\left( L^{l_{n}}T^{t_{n}}M^{m_{n}} \right)^{a_{n}}
\end{aligned}
\end{equation*}
Equate exponents!
\begin{equation*}
\begin{aligned}
	L: l_{0} &= l_{1}a_{1} + l_{2}a_{2} + \dots + l_{n}a_{n}\\
	T: t_{0} &= t_{1}a_{1} + t_{2}a_{2} + \dots + t_{n}a_{n}\\
	M: m_{0} &= m_{1}a_{1} + m_{2}a_{2} + \dots + m_{n}a_{n}\\
\end{aligned}
\end{equation*}
\begin{equation*}
\begin{aligned}
	A = \left[ \begin{array}{cccc}
		l_{1} & l_{2} & \dots & l_{n}\\
		t_{1} & t_{2} & \dots & t_{n}\\
		m_{1} & m_{2} & \dots & m_{n}\\
	\end{array} \right]_{3\times{}n}\ \ \
	\vec{x} = \left[ \begin{array}{c}
		a_{1}\\
		a_{2}\\
		\vdots\\
		a_{n}\\
	\end{array} \right]_{n\times1}\ \ \
	\vec{b} = \left[ \begin{array}{c}
		l_{0}\\
		t_{0}\\
		m_{0}\\
	\end{array} \right]_{3\times1}
\end{aligned}
\end{equation*}
In general, $n>3$ -- undetermined $\Rightarrow\infty$ set of solutions.

\[ A\vec{x} = \vec{b} \mbox{ has a solution of form}\]
\begin{equation*}
\begin{aligned}
	\vec{x} = \vec{x_{p}} + \gamma_{1}\vec{x_{1}} + \gamma_{2}\vec{x_{2}} + \dots + \gamma_{k}\vec{x_{k}} % TODO: Find out where k comes from
\end{aligned}
\end{equation*}
where $\gamma$ is arbitrary, $\vec{x_{p}}$ is the particular solution, and $\{ \vec{x_{j}} \}$ is the null space of $A$.
\begin{itemize}
	\item $\vec{x_{j}}$ correspond to exponents that yield dimensionless quantities
	\[ \mbox{\textcolor{red}{defines:} } \Pi_{j}, j=1,\dots,k \]
	\item $\vec{x_{p}}$ exponents that yield a quantity with $[q]$, call it $Q$
\end{itemize}
So
\begin{equation*}
\begin{aligned}
	q &= QF(\Pi_{1}, \Pi_{1}, \dots, \Pi_{k})\\
	\pi_{0} = \frac{q}{Q} &= QF(\Pi_{1}, \Pi_{1}, \dots, \Pi_{k})\\
\end{aligned}
\end{equation*}
where the second equation is all in terms of dimensionless quantities

\begin{example}[Drag on a sphere]
	Drag on a sphere within a fluid.
	The sphere has radius $R$ moving at velocity $v$ with a resistant drag force $D_{F}$.
	What factors play a role?
	\begin{description}
		\item[$R=$] radius $[L]$
		\item[$v=$] velocity $[LT^{-1}]$
		\item[$\rho=$] density $[ML^{-3}]$
		\item[$\mu=$] dynamic viscosity $[ML^{-1}T^{-1}]$
	\end{description}
	\begin{equation*}
	\begin{aligned}
		D_{F} &= f(R, v, \rho, \mu)\\
		[D_{F}] &= [R]^{a}[v]^{b}[\rho]^{c}[\mu]^{d}\\
		MLT^{-2} &= L^{a}\left( LT^{-1} \right)^{b}\left( ML^{-3} \right)^{c}\left( ML^{-1}T^{-1} \right)^{d}\\
		MLT^{-2} &= L^{a}\times L^{b}T^{-b}\times M^{c}L^{-3c} \times M^{d}L^{-d}T^{-d}\\
		MLT^{-2} &= M^{c+d}L^{a+b-3c-d}T^{-b-d}\\
	\end{aligned}
	\end{equation*}
	\begin{equation*}
	\begin{aligned}
		\left[ \begin{array}{rrrr}
			1 & 1 & -3 & -1\\
			0 & -1 & 0 & -1\\
			0 & 0 & 1 & 1\\
		\end{array} \right]\left[ \begin{array}{c}
			a\\
			b\\
			c\\
			d\\
		\end{array} \right] = \left[ \begin{array}{r}
			1\\
			-2\\
			1\\
		\end{array} \right]
	\end{aligned}
	\end{equation*}
	\begin{equation*}
	\begin{aligned}
		\left[ \begin{array}{rrrr|r}
			1 & 1 & -3 & -1 & 1\\
			0 & -1 & 0 & -1 & -2\\
			0 & 0 & 1 & 1 & 1\\
		\end{array} \right]
		&= \left( r_{2} \gets -r_{2} \right) \left[ \begin{array}{rrrr|r}
			1 & 1 & -3 & -1 & 1\\
			0 & 1 & 0 & 1 & 2\\
			0 & 0 & 1 & 1 & 1\\
		\end{array} \right]\\
		&= \left( r_{1} \gets r_{1} - r_{2} \right) \left[ \begin{array}{rrrr|r}
			1 & 0 & -3 & -2 & -1\\
			0 & 1 & 0 & 1 & 2\\
			0 & 0 & 1 & 1 & 1\\
		\end{array} \right]\\
		&= \left( r_{1} \gets r_{1} + 3r_{3} \right) \left[ \begin{array}{rrrr|r}
			1 & 0 & 0 & 1 & 2\\
			0 & 1 & 0 & 1 & 2\\
			0 & 0 & 1 & 1 & 1\\
		\end{array} \right]\\
		\end{aligned}
	\end{equation*}
	\begin{equation*}
	\begin{aligned}
		a + d &= 2 \sep b + d &= 2 \sep c + d &= 1\\
		a &= 2 - d \sep b &= -d + 2 \sep c &= -d + 1\\
	\end{aligned}
	\end{equation*}
	\begin{equation*}
	\begin{aligned}
		\left[ \begin{array}{c}
			a\\
			b\\
			c\\
			d\\
		\end{array} \right]
		&= \left[ \begin{array}{c}
			2 - d\\
			2 - d\\
			1 - d\\
			d\\
		\end{array} \right]\\
		&= \left[ \begin{array}{c}
			2\\
			2\\
			1\\
			0\\
		\end{array} \right]
		+ d\left[ \begin{array}{c}
			-1\\
			-1\\
			-1\\
			1\\
		\end{array} \right]\\
		&= R^{2}v^{2}\rho^{1} + R^{-1}v^{-1}\rho^{-1}\mu^{1}\\
		&= R^{2}v^{2}\rho + \frac{\mu}{Rv\rho}\\
		&= \left[ \begin{array}{c}
			2\\
			2\\
			1\\
			0\\
		\end{array} \right]
		+ \textcolor{red}{-d}\left[ \begin{array}{c} % TODO: Find out why we made d negative
			1\\
			1\\
			1\\
			-1\\
		\end{array} \right]\\
		&= R^{2}v^{2}\rho^{1} + R^{1}v^{1}\rho^{1}\mu^{-1}\\
		&= R^{2}v^{2}\rho + \frac{Rv\rho}{\mu}\\
	\end{aligned}
	\end{equation*}
	\[ D_{F} = R^{2}v^{2}\rho F\left( \frac{Rv\rho}{\mu} \right) \]
	where $\frac{Rv\rho}{\mu}$ represents the Reynold's number $Re$
\end{example}

\section{Reynold's number}\label{sec:reynold's-number}
\begin{equation}
	Re = \frac{Rv\rho}{\mu} = \frac{\mbox{inertial}}{\mbox{viscous}}
\label{eq:reynolds-number}
\end{equation}
When $Re$ is small, there will be laminar (smooth) flow, when $Re$ is large, the flow will be turbulent (having vortices).

\begin{example}[T-Rex Top Speed]
	How fast can a Tyrannosaurus Rex (T-Rex) walk or run?\\
	Develop a model!\\
	Find $v=$ velocity $[v] = \left[ \frac{L}{T} \right]$\\
	What variables are important?
	\begin{description}
		\item[$m$] mass, $[M]$
		\item[$g$] gravity, $[LT^{-2}]$
		\item[$h$] hip length, [$L$]
		\item[$s$] stride length -- distance between 2 steps, $[s]=L$
	\end{description}
	\[ v = f(s, h, g, m) \]
	\begin{equation*}
	\begin{aligned}
		[v] &= [s^{a}h^{b}g^{c}m^{d}]\\
		LT^{-1} &= L^{a}L^{b}\left( LT^{-2} \right)^{c}M^{d}\\
		LT^{-1} &= L^{a+b}L^{c}T^{-2c}M^{d}\\
		LT^{-1}M^{0} &= L^{a+b+c}T^{-2c}M^{d}\\
	\end{aligned}
	\end{equation*}
	Equate exponents:
	\begin{equation*}
	\begin{aligned}
		L:&& 1 &= a + b + c\\
		T{:}&& -1 &= -2c\\
		M:&& 0 &= d \mbox{ (independent of mass)}\\
	\end{aligned}
	\end{equation*}
	\begin{equation*}
	\begin{aligned}
		\left[ \begin{array}{rrr}
			1 & 1 & 1\\
			0 & 0 & -2
		\end{array} \right]\left[ \begin{array}{c}
			a\\
			c\\
		\end{array} \right] = \left[ \begin{array}{r}
			1\\
			-1\\
		\end{array} \right]
	\end{aligned}
	\end{equation*}
	\begin{equation*}
	\begin{aligned}
		\left[ \begin{array}{rrr|r}
			1 & 1 & 1 & 1\\
			0 & 0 & -2 & -1
		\end{array} \right] = \left( r_{2} \gets -\frac{r_{2}}{2} \right)
		\left[ \begin{array}{rrr|r}
			1 & 1 & 1 & 1\\
			0 & 0 & 1 & \frac{1}{2}
		\end{array} \right]
	\end{aligned}
	\end{equation*}
	\begin{equation*}
	\begin{aligned}
		a + b + c &= 1 \sep b = \mbox{ free } \sep c = \frac{1}{2}\\
		a + b + \frac{1}{2} &= 1 \sep b = \mbox{ free } \sep c = \frac{1}{2}\\
		a &= -b + \frac{1}{2} \sep b = \mbox{ free } \sep c = \frac{1}{2}\\
	\end{aligned}
	\end{equation*}\begin{equation*}
	\begin{aligned}
		\left[ \begin{array}{c}
			a\\
			b\\
			c\\
		\end{array} \right]
		&= \left[ \begin{array}{c}
			\frac{1}{2} - b\\
			b\\
			\frac{1}{2}\\
		\end{array} \right]\\
		&= \left[ \begin{array}{c}
			\frac{1}{2}\\
			0\\
			\frac{1}{2}\\
		\end{array} \right]
		+ b\left[ \begin{array}{c}
			-1\\
			1\\
			0\\
		\end{array} \right]\\
		&= s^{\frac{1}{2}}g^{\frac{1}{2}} + s^{-1}h^{1}\\ % TODO: Again, the free variable was multipled by -1, find out why
		&= \sqrt{sg} + \frac{h}{s}\\
	\end{aligned}
	\end{equation*}
	\begin{equation*}
	\begin{aligned}
		b\left[ \begin{array}{c}
			-1\\
			1\\
			0\\
		\end{array} \right] \Rightarrow
		(-1)\left[ \begin{array}{c}
			-1\\
			1\\
			0\\
		\end{array} \right] \Rightarrow
		\left[ \begin{array}{c}
			1\\
			-1\\
			0\\
		\end{array} \right] \Rightarrow s^{1}h^{-1} \Rightarrow \frac{s}{h}
	\end{aligned}
	\end{equation*}
	\[ v = \sqrt{gh}f\left( \frac{s}{h} \right) \] % TODO: Check if the notes had a mistake putting \sqrt{gh} instead of \sqrt{sg}
\end{example}

\section{Froude number}\label{sec:froude-number}
As found in the previous example:
\begin{equation*}
	\frac{v}{\sqrt{gh}} = \frac{\mbox{inertial force}}{\mbox{gravitational force}}
	\label{eq:froude-number}
\end{equation*}

\end{document}