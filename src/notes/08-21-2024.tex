%! Author = Len Washington III
%! Date = 8/21/24

% Preamble
\documentclass[
	date={August 21{,} 2024}
]{math486notes}

% Document
\begin{document}

\begin{example}
	\[ y'' + 2y' + y = 5 \]
	Assume $y=e^{mx}$
	\begin{equation*}
	\begin{aligned}
		e^{mx} + 2e^{mx} + e^{mx} &= 5\\
		m^{2}e^{mx} + 2me^{mx} + e^{mx} &= 5\\
		m^{2}e^{mx} + 2me^{mx} + e^{mx} &= 5\\
	\end{aligned}
	\end{equation*}

	Find the auxiliary function:
	\[ y'' + 2y' + y = 0 \]
	\begin{equation*}
		\begin{aligned}
			m^{2}e^{mx} + 2me^{mx} + e^{mx} &= 0\\
			e^{mx}\left(m^{2} + 2m + 1\right) &= 0\\
			m^{2} + 2m + 1 &= \frac{0}{e^{mx}}\\
			m^{2} + 2m + 1 &= 0\\
			(m+1)^{2} &= 0\\
		\end{aligned}
	\end{equation*}
	\begin{equation*}
	\begin{aligned}
		m_{1}+1 &= 0\ \ & \ \ m_{2} + 1 &= 0\\
		m_{1} &= -1\ \ & \ \ m_{2} &= -1\\
		y_{1} &= e^{-1x}\ \ & \ \ y_{2} &= xe^{-1x}\\
		y_{1} &= e^{-x}\ \ & \ \ y_{2} &= xe^{-x}\\
	\end{aligned}
	\end{equation*}
\end{example}

\begin{example}
	\begin{equation*}
	\begin{aligned}
		x_{1} - 2x_{2} &= 4\\
		\dots&
	\end{aligned}
	\end{equation*}
\end{example}

\section{Parameter estimation}\label{sec:parameter-estimation}
\[ \mbox{Data: } (x_{i}, y_{x})^{N}_{i=1} \]

Estimate $c_{1}, c_{2}$ in the equation of a line $c_{1} + c_{2}x = y$.
2 unknowns and $N$ data points.
Use points.
\[ c_{1} + c_{2}x_{i} = y_{i};\ i = 1\dots{}N \]
Matrix formulation:
\[ \left[ \begin{array}{cc}
	1 & x_{1}\\
	1 & x_{2}\\
	\vdots & \vdots\\
	1 & x_{n}\\
\end{array} \right] \]

\begin{example}
	\begin{equation*}
	\begin{aligned}
		[A | b] &= \left[ \begin{array}{ccc|c}
			1 & 2 & 1 & 1\\
			2 & 4 & 2 & 2\\
		\end{array} \right]\\
	&= (r_{2} \gets r_{2} - 2r_{1})\left[ \begin{array}{ccc|c}
			1 & 2 & 1 & 1\\
			0 & 0 & 0 & 0\\
		\end{array} \right]
	\end{aligned}
	\end{equation*}
	$x_{1}$ is a lead variable, $x_{2}$ and $x_{3}$ are free variables.
	Assume $x_{2}=b$ and $x_{3}=a$.
	\begin{equation*}
	\begin{aligned}
		x_{3} &= a\\
		x_{2} &= b\\
		x_{1} &= 1 - 2b - a\\
	\end{aligned}
	\end{equation*}
	\begin{equation*}
	\begin{aligned}
		\vec{x} &= \left[ \begin{array}{c}
			1 - 2b - a\\
			b\\
			a\\
		\end{array} \right]\\
		&= \left[ \begin{array}{c}
			1\\
			0\\
			0\\
		\end{array} \right] + a\left[ \begin{array}{c}
			-1\\
			0\\
			1\\
		\end{array} \right] + b\left[ \begin{array}{c}
			-2\\
			1\\
			0\\
		\end{array} \right]\\
	\end{aligned}
	\end{equation*}
	\[ N(A) = \mbox{null space/kernel} = \left\{ \left[ \begin{array}{c}
		-1\\
		0\\
		1\\
	\end{array} \right], \left[ \begin{array}{c}
		-2\\
		1\\
		0\\
	\end{array} \right] \right\} \]

	\textbf{Note: The particular solution $\left[ \begin{array}{c}
		1\\
		0\\
		0\\
	\end{array} \right]$ is not unique}
\end{example}

\definition{rank($A$)}{\# of independent rows.}
$\Call{Rank}{A} + \Call{Dim}{N(A)} = $ \# of columns

\section{Dimensional Analysis}\label{sec:dimensional-analysis}
3 Aspects
\begin{enumerate}[label=(\arabic*)]
	\item \definition{Dimensional reduction}{develop functional laws w/ reduced number of variables and parameters}
	\item Determine invariance
	\item Scaling or non-dimensionalize equations (important for computing)
\end{enumerate}

\subsection{Key Idea}\label{subsec:key-idea}
Any ``physical'' quantity can be described in terms of fundamental properties such as mass, length, time, temperature, current, ``price'', etc.
These properties are called \textbf{dimensions} or \textbf{fundamental dimensions.}

\subsection{Scale Units}\label{subsec:scale-units}
\definition{Scale Units}{assign a number to the dimension - number depends on units}

\subsubsection{Mass Dimension $M$}\label{subsubsec:mass-dimension-$m$}
Units:
\begin{description}
	\item[CGS] grain
	\item[MKS] kilogram
	\item[FPS] slug
	\item[SI] kilogram
\end{description}

\subsubsection{Length Dimension $L$}\label{subsubsec:length-dimension-$l$}
Units:
\begin{itemize}
	\item 1'
	\item 12''
	\item 0.3048m
	\item 30.48cm
	\item 0.000189 miles
	\item $3.048\times10^{8}$ nanometers
\end{itemize}

\noindent Dimensions are independent of scale units.

\noindent \definition{Fundamental Dimensions}{the basic or primary building blocks of other physical quantities (secondary or derived) within the same system.}

\begin{table}[H]
	\centering
	\caption{Dimensions}
	\label{tab:dimensions}
	\begin{tabular}{lcl}
		\textbf{Dimension} & \textbf{Symbol} & \textbf{SI Unit}\\
		Mass & M & kg\\
		Length & L & m (meter)\\
		Time & T & s (sec)\\
		Temperature & $\theta$ & K (kelvin)\\
		Current & I & A (ampere)\\
		Light & C & cd (candela)
	\end{tabular}
\end{table}

Derived Quantities:
\begin{equation}
	[\mbox{force}] = M\frac{L}{T^{2}}\label{eq:force-quantity}
\end{equation}

\subsection{Laws}\label{subsec:laws}
\begin{itemize}
	\item Physical Laws must be independent of scale eunits (must hole for \emph{any} unit)
	\item A quantity $Y$ is \emph{dimensionless} if $[Y]=1=M^{0}L^{0}T^{0}$
\end{itemize}

\begin{example}[{What is $[\theta]$ if $\theta$ is in radians?}]
	$[\theta]=1$ because degrees measured in radians are $\theta=\frac{s}{r}$.
	The units for both $s$ and $r$ is L ($[s]=L, [r]=L$).
	\[ [\theta] = \left[ \frac{s}{r} \right] = \frac{L}{L} = 1 \]
\end{example}

\end{document}